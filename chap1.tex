\chapter{Introduction}

California's public higher education system is a titan of American academia, a complex, three-tiered structure comprising the University of California (UC), California State University (CSU), and the California Community Colleges (CCC)~\cite{ppic}. Collectively, these 149 colleges and universities serve nearly 2.9 million students, forming the largest public higher education system in the United States~\cite{ppic,uc,calstate,cccco}. A foundational principle of this system is the promise of student mobility, particularly the pathway from a two-year community college to a four-year university~\cite{ppic}.

However, the mechanism designed to facilitate this movement—the process of determining course equivalency, or articulation, is a formidable, largely manual process that creates significant barriers for students~\cite{pardos2019}. At the heart of this process is the Articulation System Stimulating Interinstitutional Student Transfer (ASSIST), the state's official public repository for articulation agreements~\cite{assistinfo}. While ASSIST provides a centralized platform for students and advisors to view established equivalencies, it is fundamentally a display for agreements that are negotiated and updated manually by articulation officers at each individual campus~\cite{assistfaq}. Given the sheer number of institutions, defining and maintaining these agreements is a task of bleak combinatorics, rendering it inefficient, slow, and inherently intractable~\cite{pardos2019}. This manual paradigm places a considerable burden on academic advisors and administrative staff, who must meticulously review course descriptions and syllabi to compare content, rigor, and learning outcomes~\cite{pardos2019}. The result is a system that struggles to keep pace with the needs of a vast and mobile student body, making California a critical case study for a problem that extends far beyond its borders.

The challenges exemplified by California's system are a microcosm of a systemic problem in American higher education~\cite{collegeopportunity2017}. Transferring between institutions has become a normative component of the modern student's academic journey~\cite{publicagenda2025}. Data from the National Student Clearinghouse Research Center reveals that in the fall of 2023, transfer enrollment constituted 13.2\% of all continuing and returning undergraduates~\cite{nscnews2023}. This trend represents a post-pandemic resurgence in student mobility, with transfer enrollment growing by 5.3\% from Fall 2022 to Fall 2023 and an additional 4.4\% in Fall 2024~\cite{nscnews2023,nscnews20250305}. This mobile population is increasingly diverse, comprising not only students following traditional two-year to four-year pathways but also a substantial number of returning learners~\cite{nscdd20250507}. Over half of these returning students opt to re-enroll at a new institution, underscoring the critical role of the transfer system in providing flexible pathways to degree completion~\cite{nscdd20250507}.

The consequences of this systemic inefficiency are borne almost entirely by students, manifesting in significant academic and financial setbacks~\cite{gao2017}. The most direct outcome is the loss of earned academic credit. A 2017 report by the U.S. Government Accountability Office (GAO) estimated that students who transferred between 2004 and 2009 lost an average of 43\% of their credits~\cite{gao2017}. This finding is echoed across studies, with reports indicating that more than half of all transfer students lose some credits, and approximately one-fifth are forced to repeat courses for which they have already received a passing grade~\cite{publicagenda2025}.

This credit loss creates a cascade of negative consequences. It invariably increases the time-to-degree, delaying graduation and entry into the workforce~\cite{gao2017}. Each repeated course also carries a financial cost, increasing the total tuition burden and potentially exhausting a student's eligibility for federal financial aid~\cite{gao2017}. A process often undertaken to save money can paradoxically result in a greater overall financial commitment~\cite{collegeopportunity2017}. The frustration inherent in the transfer process also has a measurable impact on student persistence, with transfer students tending to have lower graduation rates than their non-transfer peers~\cite{porter1999}.

This issue transcends mere administrative inefficiency and becomes a critical matter of educational equity~\cite{collegeopportunity2017}. Low-income students and students from historically underrepresented racial and ethnic groups are more likely to begin at community colleges and rely on transfer pathways~\cite{ace2025}. Recent growth in transfer enrollment has been driven disproportionately by Black and Hispanic students~\cite{nscnews2023}. Therefore, the barriers imposed by an inefficient articulation system disproportionately harm the very student populations that institutions are striving to support. A clear and troubling feedback loop emerges: the manual nature of course articulation causes credit loss, which imposes academic and financial burdens that fall most heavily on underrepresented students, undermining institutional goals of closing equity gaps. The seemingly low-level administrative task of determining course equivalency is thus revealed to be a significant driver of systemic inequity in higher education.

Addressing this challenge through robust automation is not merely an operational optimization; it is a necessary intervention to foster a more equitable, efficient, and supportive educational ecosystem. This thesis confronts this problem by developing and validating a novel computational framework to automate course articulation. By leveraging deep metric learning on publicly available course catalog text, this research introduces a privacy-preserving and scalable method that decouples semantic representation from classification.

The primary contributions of this work are the development of a highly accurate automated framework, an innovative feature engineering technique that improves classification, and a computationally efficient approach that avoids the privacy and scalability issues of previous methods. The remainder of this thesis will detail the methodology, experiments, and results of this approach.

\section{Thesis Roadmap}
The remainder of this thesis is structured to provide a comprehensive account of this research.
\begin{itemize}
    \item \textbf{Chapter 2: Background and Related Work} will provide a detailed in-depth survey of the landscape of student transfer automation and the evolution of technological interventions.
    \item \textbf{Chapter 3: Methodology} will offer a deep dive into the data collection and preparation processes, the specific embedding models evaluated, the construction of the feature vectors, and the theoretical underpinnings of the machine learning classifiers employed.
    \item \textbf{Chapter 4: Experimental Setup and Results} will detail the experimental design, the datasets used for training and validation, and a comprehensive analysis of the classification performance, including ablation studies and model comparisons.
    \item \textbf{Chapter 5: Discussion and Future Work} will interpret the results in a broader context, discuss the limitations of the current study, and outline promising avenues for future research, including the development of a full-scale course recommendation system and the exploration of fine-tuning techniques.
    \item \textbf{Chapter 6: Conclusion} will summarize the key findings of the thesis and reiterate the significance of its contributions to both academic research and the practical administration of higher education.
\end{itemize}