\chapter{
    Introduction
    }

California's public higher education system is a titan of American academia, a complex, three-tiered structure comprising the University of California (UC), California State University (CSU), and the California Community Colleges (CCC)~\cite{ppic}. Collectively, these 149 colleges and universities serve nearly 2.9 million students, forming the largest public higher education system in the United States~\cite{ppic,uc,calstate,cccco}. A foundational principle of this system is the promise of student mobility, particularly the pathway from a two-year community college to a four-year university. However, the mechanism designed to facilitate this movement, the process of determining course equivalency, or articulation, is a formidable, largely manual process that creates significant barriers for students.

At the heart of this process is the Articulation System Stimulating Interinstitutional Student Transfer (ASSIST), the state's official public repository for articulation agreements~\cite{assistinfo}. While ASSIST provides a centralized platform for students and advisors to view established equivalencies, it is fundamentally a display for agreements that are negotiated and updated manually by articulation officers at each individual campus~\cite{assistfaq}. Given the sheer number of institutions, the process of defining and maintaining these agreements is a task of bleak combinatorics, rendering it inefficient, slow, and inherently intractable~\cite{pardos2019}. This manual paradigm places a considerable burden on academic advisors and administrative staff, who must meticulously review course descriptions and syllabi to compare content, rigor, and learning outcomes~\cite{pardos2019}. The result is a system that struggles to keep pace with the needs of a vast and mobile student body, making California a critical case study for a problem that extends far beyond its borders.

The challenges exemplified by California's system are a microcosm of a systemic crisis in American higher education. The act of transferring between institutions has become a normative component of the modern student's academic journey. Data from the National Student Clearinghouse Research Center reveals that in the fall of 2023, transfer enrollment constituted 13.2\% of all continuing and returning undergraduates~\cite{nscnews2023}. This trend is not static; it represents a post-pandemic resurgence in student mobility, with transfer enrollment growing by 5.3\% from Fall 2022 to Fall 2023 and an additional 4.4\% in Fall 2024~\cite{nscnews2023,nscnews20250305}. This mobile population is increasingly diverse, comprising not only students following traditional two-year to four-year pathways but also a substantial number of returning learners who have previously paused their education. Over half of these returning students opt to re-enroll at a new institution, underscoring the critical role of the transfer system in providing flexible pathways to degree completion~\cite{nscdd20250507}.

The consequences of this systemic inefficiency are borne almost entirely by the students, manifesting in significant academic and financial setbacks. The most direct and damaging outcome is the loss of earned academic credit. A comprehensive 2017 report by the U.S. Government Accountability Office (GAO) estimated that students who transferred between 2004 and 2009 lost an average of 43\% of their credits in the process~\cite{gao2017}. This finding is echoed across numerous studies, with reports indicating that more than half of all transfer students lose at least some credits, and approximately one-fifth are forced to repeat courses for which they have already received a passing grade at a previous institution~\cite{publicagenda2025}.

This loss of credit creates a cascade of negative consequences. It invariably leads to an increased time-to-degree, delaying graduation and entry into the workforce. Each repeated course also carries a financial cost, increasing the total tuition burden and potentially exhausting a student's eligibility for federal financial aid programs like Pell Grants and Direct Loans~\cite{gao2017}. A process that is often undertaken to save money (for example, by starting at a less expensive community college) can paradoxically result in a greater overall financial commitment, trapping students in a cycle of additional coursework and debt~\cite{collegeopportunity2017}.

The friction and frustration inherent in the transfer process also have a measurable impact on student persistence and graduation. Studies have shown that transfer students, as a group, tend to have lower retention and graduation rates than their peers who begin and end their studies at the same institution~\cite{porter1999}. This issue transcends mere administrative inefficiency and becomes a critical matter of educational equity. Low-income students and students from historically underrepresented racial and ethnic groups are more likely to begin their postsecondary journey at community colleges and rely on transfer pathways to attain a bachelor's degree~\cite{ace2025}. The recent growth in transfer enrollment has been driven disproportionately by Black and Hispanic students~\cite{nscnews2023}. Therefore, the barriers imposed by an inefficient articulation system such as credit loss, increased cost, and delayed graduation, disproportionately harm the very student populations that institutions are striving to support.

A clear and troubling feedback loop emerges from this analysis. The fundamentally manual and inefficient nature of course articulation is a direct cause of credit loss. This credit loss imposes a tangible academic and financial burden on students which falls most heavily on underrepresented and low-income students, who are a large and growing segment of the transfer population. This disproportionate impact, in turn, undermines institutional goals of improving student retention and closing persistent equity gaps in degree attainment. Thus, the seemingly low-level administrative task of determining course equivalency is revealed to be a significant driver of systemic inequity in higher education. Addressing this challenge through robust automation is not merely an operational optimization; it is a necessary intervention to foster a more equitable, efficient, and supportive educational ecosystem for all students.

\section{Contributions}
This research makes several key contributions to the fields of educational data mining and natural language processing, offering a practical and powerful solution to the long-standing challenge of course articulation.
\begin{itemize}
\item \textbf{A High-Accuracy, Automated Framework}: This thesis develops and validates a novel framework for determining course equivalency that achieves state-of-the-art accuracy, with \(F_1\)-scores exceeding 0.99 on a challenging real-world dataset. Crucially, it accomplishes this using only publicly available course catalog text, making it broadly applicable.
\item \textbf{An Innovative Feature Engineering Technique}: It introduces a composite distance vector, \(\Delta_c\), that uniquely combines element-wise embedding differences with cosine similarity. This technique provides a richer input signal for classification and is shown to demonstrably improve the performance of downstream machine learning models, particularly linear classifiers.
\item \textbf{A Computationally Efficient and Scalable Approach}: The research demonstrates that by decoupling semantic representation from classification, it is possible to harness the power of deep contextual embeddings without the high computational costs, API dependencies, and opaque nature of direct LLM-based classification. This makes the proposed solution more efficient, scalable, and practical for institutional deployment.
\item \textbf{A Privacy-Preserving Methodology}: By relying exclusively on public course descriptions, the proposed method circumvents the significant privacy, security, and data access challenges associated with techniques that require sensitive student enrollment records. This makes the framework more ethically sound and generalizable across any pair of institutions, regardless of their data-sharing agreements.
\end{itemize}

\section{Thesis Roadmap}
The remainder of this thesis is structured to provide a comprehensive account of this research.
\begin{itemize}
\item \textbf{Chapter 2: Background and Related Work} will provide a detailed in-depth survey of the landscape of student transfer automation and the evolution of technological interventions.
\item \textbf{Chapter 3: Methodology} will offer a deep dive into the data collection and preparation processes, the specific embedding models evaluated, the construction of the feature vectors, and the theoretical underpinnings of the machine learning classifiers employed.
\item \textbf{Chapter 4: Experimental Setup and Results} will detail the experimental design, the datasets used for training and validation, and a comprehensive analysis of the classification performance, including ablation studies and model comparisons.
\item \textbf{Chapter 5: Discussion and Future Work} will interpret the results in a broader context, discuss the limitations of the current study, and outline promising avenues for future research, including the development of a full-scale course recommendation system and the exploration of fine-tuning techniques.
\item \textbf{Chapter 6: Conclusion} will summarize the key findings of the thesis and reiterate the significance of its contributions to both academic research and the practical administration of higher education.
\end{itemize}