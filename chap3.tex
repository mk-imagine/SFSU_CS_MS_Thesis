\chapter{Methodology}

\section{Data Description}

% When you describe your implementation, you may need to use code or pseudocode to help convey your point.

% The lstlisting example shown in~\ref{listing:stencil-core} can be useful.

% \begin{lstlisting}[caption={Stencil computation in 2D: performs sum of product of nearby pixels with weights.},label={listing:stencil-core}, name=stencil-core, float=h, style=mystyle,language=C++]
% float smoothPixel(Si, Sj, S, R, weights) {
%     // compute the weight sum of pixels nearby
%     // this code doesn't handle edge conditions
%     // and assumes sum of weights[i,j] = 1.0
%     float sum = 0.0;
%     for (int j=0; j<R; j++)
%         for (int i=0; i<R; i++)
%            sum += weights[i,j]*S[Si+i,Sj+j]
%     return sum; }
% \end{lstlisting}


